\documentclass{article}
\usepackage{fancyhdr}
\usepackage{multicol}
\usepackage[margin=1in]{geometry} % Ajuste as margens aqui
\usepackage{amsmath} % Pacote para fórmulas matemáticas

\pagestyle{fancy}
\fancyhf{}
\chead{Física Experimental II • Experimento realizado em 12/05/2023}
\rfoot{\thepage}
\renewcommand{\headrulewidth}{0.4pt}

\begin{document}

\begin{center} % Inicia um ambiente de alinhamento central
    \huge % Define o tamanho da fonte como "Huge" (enorme)
    \textbf{Determinação da densidade de um sólido} 
    \normalsize
    \vspace{10pt}
    \\
    Matheus Aparecido Souza Silva, João Vitor Costa, Isabela Santana, Gustavo Pereira
    \\
    \vspace{5pt}
    \textbf{Turma}: TA \textbf{Horário}: 6M23  \textbf{Curso}: Engenharia Elétrica
\end{center} 

\begin{multicols*}{2} % Inicia o ambiente de duas colunas, permitindo a quebra de página

    \section{Metodologia}
    A metodologia utilizada visou determinar a densidade
    de um cilindro contendo um furo circular no centro de
    sua base, por meio de três modelos experimentais distintos.
    \subsection{Modelo 1}
    O primeiro modelo adotado foi: 

    \begin{equation}
        \rho_1 = \frac{m}{V},
    \end{equation} no qual $m$ representa a massa do sólido, determinada diretamente utilizando uma balança semi-analítica de resolução 1g e $V$
    representa o volume de água deslocada quando o cilindro é inserido em um béquer graduado com resolução de 5mL

    \subsection{Modelo 2}
    O terceiro modelo adotado foi:  
    \begin{equation}
        \rho_3 = \frac{m}{m_a} \rho a,
    \end{equation}  no qual $m$ e $m_{\text{a}}$ representam a massa do sólido e a massa de água contida no béquer, respectivamente, determinadas diretamente utilizando uma balança semianalítica de resolução 1 g. $\rho_{a}$ representa o valor da densidade da água, que é obtida como a razão entre sua massa e seu volume.

    \begin{equation}
        \rho(m, V) = \frac{m}{V}
        \end{equation}
        O valor de referência da densidade da água é 1 g/cm\(^3\).

Neste modelo, estamos substituindo a medição direta do volume de água deslocado quando o sólido é inserido no béquer pela medição direta da respectiva massa de água deslocada quando diferentes sólidos são inseridos no béquer.

     \subsection{Modelo 3}
        O segundo modelo adotado foi:
        \begin{equation}
            \rho_2 = \frac{4m}{\pi(d_{e}^2 - d_{i}^2)h}
            \end{equation}
            No qual \(m\) representa a massa do sólido, determinada diretamente utilizando uma balança semi-analítica de resolução 1 g, \(h\) representa a altura do cilindro, e \(d_e\) e \(d_i\) representam seus diâmetros externo e interno, respectivamente. As medições de \(h\), \(d_e\), e \(d_i\) foram realizadas utilizando um paquímetro de resolução 0,05 mm.


    \section{Determinação da densidade da água}
    A densidade da água foi obtida como a razão entre sua massa e seu volume:
    \begin{equation}
    \rho(m, V) = \frac{m}{V}
    \end{equation}
    Usando a balança e o béquer, obteve-se os seguintes resultados de medição:
    \begin{equation}
        m = (248,00 \pm 0,58) \text{g e } V = (250,0 \pm 2,9) \, \text{cm}^3
    \end{equation}  A incerteza deste modelo de medição é dada por:

    \begin{align*}
        u_{\rho} &= \sqrt{\left(\frac{\partial \rho}{\partial m}\right)^2 \sigma^2_m + \left(\frac{\partial \rho}{\partial V}\right)^2 \sigma^2_V} \\
        &= \sqrt{\left(\frac{1}{\overline{V}}\right)^2 \sigma^2_m + \left(\frac{-\overline{m}}{\overline{V}^2}\right)^2 \sigma^2_V} \\
        &= \frac{\overline{m}}{\overline{V}} \sqrt{\left(\frac{\sigma_m}{\overline{m}}\right)^2 + \left(\frac{\sigma_V}{\overline{V}}\right)^2} \\
        &= \frac{248}{250} \sqrt{\left(\frac{0.58}{248}\right)^2 + \left(\frac{2.9}{250}\right)^2} \\
        &\approx 0.01168511303610994 \\
        &\approx 0.012 \, \text{g/cm}^3.
        \end{align*}
        
        Logo, o valor principal será:
\begin{equation}
\rho_{\text{média}} = \frac{\overline{m}}{\overline{V}} = 0.992 \, \text{g/cm}^3.
\end{equation}

Finalmente:
\begin{equation}
\rho = (0.992 \pm 0.012) \, \text{g/cm}^3.
\end{equation}

        
    \subsection{Modelo 1}
    Este é o conteúdo da Subseção 2.2.
    
    \end{multicols*} % Fecha o ambiente de duas colunas
    
    \end{document}
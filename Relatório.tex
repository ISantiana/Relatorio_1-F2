\documentclass{article}
\usepackage{fancyhdr}
\usepackage{multicol}
\usepackage[margin=1in]{geometry}
\usepackage{amsmath}
\usepackage{siunitx}
\usepackage{enumerate}

\pagestyle{fancy}
\fancyhf{}
\chead{Física Experimental II • Experimento realizado em 27/10/2023}
\rfoot{\thepage}
\renewcommand{\headrulewidth}{0.4pt}

\begin{document}

\begin{center}
    \huge
    \textbf{Calor específico de um sólido}
    \normalsize
    \vspace{10pt}

    Matheus Aparecido Souza Silva, Isabela Sant' Ana, Gustavo Peres, João Vitor Costa
    \vspace{5pt}

    \textbf{Turma}: TA \textbf{Horário}: 6M23 \textbf{Curso}: Engenharia Elétrica
\end{center}

\begin{multicols}{2}

\section{Metodologia}

\subsection{Objetivo}

Neste relatório, descreveremos o experimento realizado para determinar o calor específico de um sólido.

\subsection{Materiais utilizados e métodos de medição}

\begin{itemize}
    \item Sólido metálico: A amostra do sólido metálico é o objeto de estudo do experimento. Sua massa é medida com uma balança.
    \item Calorímetro: O calorímetro é um dispositivo projetado para medir a quantidade de calor envolvida em uma reação ou processo físico. Ele é utilizado para conter a amostra e a água durante o experimento.
    \item Termômetro: O termômetro é usado para medir a temperatura ambiente da água no calorímetro e a temperatura da água no calorímetro após a amostra ser colocada.
    \item Fonte de calor: A fonte de calor é utilizada para aquecer a amostra metálica de forma uniforme antes de colocá-la na água do calorímetro.
    \item Balança: A balança é usada para medir a massa da amostra.
    \item Água: A água é usada como substância termométrica no calorímetro. Sua massa é medida pela balança e a temperatura pelo termômetro.
\end{itemize}

No experimento de determinação do calor específico de um sólido, estão envolvidos métodos de medição direta e indireta para obter os dados necessários para calcular o calor específico do sólido.

Métodos de medição direta: medição de massa e temperatura

Métodos de medição indireta: cálculo do calor específico do sólido.

\section{Resultados}
\subsection{Tabela de dados experimentais}

\begin{minipage}{\linewidth}
\centering
\begin{tabular}{|c|c|c|c|c|}
    \hline
    $m_1$(g) & $m_s$(g) & $T_1\,^{\circ}\mathrm{C}$    & $T_2\,^{\circ}\mathrm{C}$ &$T_{\text{eq}},^{\circ}\mathrm{C}$ \\
    \hline
    200 $\pm$ & 150 $\pm$   & 25,2 $\pm$  & 90,4 $\pm$  & 29,1 $\pm$   \\
    \hline
    200 $\pm$  & 100 $\pm$  & 25,4 $\pm$  & 90,4 $\pm$  & 28,1 $\pm$   \\
    \hline
    200 $\pm$ & 50 $\pm$   & 25,9 $\pm$  & 90,3 $\pm$  & 27,3 $\pm$    \\
    \hline
\end{tabular}
\end{minipage}

\subsection{Cálculo do calor específico do sólido a partir dos dados experimentais}
Para o cálculo do calor específico foi utilizado a formula abaixo:
\begin{equation}
    c_s = \frac{{T_1 - T_{\text{eq}}}}{{T_{\text{eq}} - T_2}} \cdot \frac{C^*}{m_s}
    \end{equation}
    
Na primeira tentativa, o cálculo do calor específico do sólido foi de 0,091 $\pm$ alguma incerteza.

Na segunda tentativa, o resultado foi de 0,093 $\pm$ alguma incerteza.

E na última tentativa, obtivemos 0,096 $\pm$ alguma incerteza.

\subsection{Média dos calores específicos}

A média simples dos valores do item 2.2 é: alguma coisa $\pm$ outra coisa (deu priguiça, calculem ai esse tbm)

\section{Discução sobre os resultados obtidos}

\end{multicols}
\end{document}

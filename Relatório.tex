\documentclass{article}
\usepackage{fancyhdr}
\usepackage{multicol}
\usepackage[margin=1in]{geometry} % Ajuste as margens aqui
\usepackage{amsmath} % Pacote para fórmulas matemáticas

\pagestyle{fancy}
\fancyhf{}
\chead{F´ısica Experimental II • Experimento realizado em 12/05/2023}
\rfoot{\thepage}
\renewcommand{\headrulewidth}{0.4pt}

\begin{document}

\begin{center} % Inicia um ambiente de alinhamento central
    \huge % Define o tamanho da fonte como "Huge" (enorme)
    \textbf{Determinação da densidade de um sólido} 
    \normalsize
    \vspace{10pt}
    \\
    Matheus Aparecido Souza Silva, João Vitor Costa, Isabela Santana, Gustavo Pereira
    \\
    \vspace{5pt}
    \textbf{Turma}: TA \textbf{Horário}: 6M23  \textbf{Curso}: Engenharia Elétrica
\end{center} 

\begin{multicols*}{2} % Inicia o ambiente de duas colunas, permitindo a quebra de página

    \section{Metodologia}
    A metodologia utilizada visou determinar a densidade
    de um cilindro contendo um furo circular no centro de
    sua base, por meio de trˆes modelos experimentais distintos.
    \subsection{Modelo 1}
    O primeiro modelo adotado foi: 

    \begin{equation}
        \rho = \frac{m}{V}
    \end{equation} no qual {m} representa a massa do sólido, determinada diretamente utilizando uma balança semi-analítica de resolução 1g e {V} 
    representa o volume de água deslocada quando o cilindro é inserido em um béquer graduado com resolução de 5mL

    \subsection{Subseção 1.2}
    Este é o conteúdo da Subseção 1.2.
    
    \section{Seção 2}
    Este é o conteúdo da Seção 2.
    
    \subsection{Subseção 2.1}
    Este é o conteúdo da Subseção 2.1.
    
    \subsection{Subseção 2.2}
    Este é o conteúdo da Subseção 2.2.
    
    \end{multicols*} % Fecha o ambiente de duas colunas
    
    \end{document}
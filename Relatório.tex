\documentclass{article}
\usepackage{fancyhdr}
\usepackage{multicol}
\usepackage[margin=1in]{geometry}
\usepackage{amsmath}
\usepackage{siunitx} 

\pagestyle{fancy}
\fancyhf{}
\chead{Física Experimental II • Experimento realizado em 20/10/2023}
\rfoot{\thepage}
\renewcommand{\headrulewidth}{0.4pt}

\begin{document}

\begin{center}
    \huge
    \textbf{Determinação da densidade de um sólido}
    \normalsize
    \vspace{10pt}

    Matheus Aparecido Souza Silva, Isabela Sant' Ana, Gustavo Peres, João Vitor Costa
    \vspace{5pt}

    \textbf{Turma}: TA \textbf{Horário}: 6M23 \textbf{Curso}: Engenharia Elétrica
\end{center}

\begin{multicols}{2}

\section{Metodologia}

A metodologia utilizada visou determinar a densidade de um cilindro contendo um furo circular no centro de sua base, por meio de três modelos experimentais distintos.

\subsection{Modelo 1}
O primeiro modelo adotado foi:

\begin{align}
\rho_1 &= \frac{m}{V},
\end{align}

onde $m$ representa a massa do sólido, determinada diretamente utilizando uma balança semi-analítica de resolução 1 g e $V$ representa o volume de água deslocada quando o cilindro é inserido em um béquer graduado com resolução de 5 mL.

\subsection{Modelo 2}
O segundo modelo adotado foi:

\begin{align}
\rho_2 &= \frac{m}{m_a} \rho_a,
\end{align}

onde $m$ e $m_a$ representam a massa do sólido e a massa de água contida no béquer, respectivamente, determinadas diretamente utilizando uma balança semianalítica de resolução 1 g. $\rho_a$ representa o valor da densidade da água, que é obtida como a razão entre sua massa e seu volume.

\begin{equation}
\rho(m, V) = \frac{m}{V}
\end{equation}

O valor de referência da densidade da água é \SI{1}{\gram\per\centi\meter\cubed}. Neste modelo, estamos substituindo a medição direta do volume de água deslocado quando o sólido é inserido no béquer pela medição direta da respectiva massa de água deslocada quando diferentes sólidos são inseridos no béquer.

\subsection{Modelo 3}
O terceiro modelo adotado foi:

\begin{align}
\rho_2 &= \frac{4m}{\pi(d_e^2 - d_i^2)h},
\end{align}

onde $m$ representa a massa do sólido, determinada diretamente utilizando uma balança semi-analítica de resolução 1 g, $h$ representa a altura do cilindro, e $d_e$ and $d_i$ representam seus diâmetros externo e interno, respectivamente. As medições de $h$, $d_e$, e $d_i$ foram realizadas utilizando um paquímetro de resolução 0.05 mm.

\section{Resultados}

\subsection{Determinação da densidade da água}
A densidade da água foi obtida como a razão entre sua massa e seu volume:

\begin{align}
\rho(m, V) &= \frac{m}{V}
\end{align}

Usando a balança e o béquer, obtiveram-se os seguintes resultados de medição:

\begin{align}
m &= (239.00 \pm 0.58) \, \text{g e } V = (250.0 \pm 2.9) \, \text{cm}^3
\end{align}

A incerteza deste modelo de medição é dada por:

\begin{align}
u_{\rho} &= \sqrt{\left(\frac{\partial \rho}{\partial m}\right)^2 \sigma^2_m + \left(\frac{\partial \rho}{\partial V}\right)^2 \sigma^2_V} \\
\end{align}

Porém como a incerteza da massa é desprezível em relação à incerteza do volume o novo modelo do cálculo da incerteza é dado por:

\begin{align}
u_{\rho} &= \sqrt{\left(\frac{\partial \rho}{\partial V}\right)^2 \sigma^2_V} \\
&= \sqrt{\left(\frac{-\overline{m}}{\overline{V}^2}\right)^2 \sigma^2_V} \nonumber \\
&= \frac{\overline{m}}{\overline{V}} \sqrt{\left(\frac{\sigma_V}{\overline{V}}\right)^2} \nonumber \\
&= \frac{239}{250} \sqrt{\left(\frac{2.9}{250}\right)^2} \nonumber \\
&\approx 0.011 \nonumber
\end{align}

Logo, o valor principal será:

\begin{align}
\rho_{\text{média}} &= \frac{\overline{m}}{\overline{V}} = 0.956 \, \text{g/cm}^3.
\end{align}

Finalmente:

\begin{align}
\rho &= (0.956 \pm 0.011) \, \text{g/cm}^3.
\end{align}

\subsection{Modelo 1}
Os valores encontrados para a massa $m$ e o volume $V$ referentes ao primeiro modelo experimental foram:

\begin{align}
m &= (300.00 \pm 0.58) \, \text{g} \quad \text{e} \quad V = (30.0 \pm 2.9) \, \text{cm}^3.
\end{align}

Para ambas as medições, o teste de flutuação de resultados indicou incertezas do tipo B. Assumindo distribuições de probabilidades retangulares para as duas medições, as incertezas foram calculadas de acordo com as expressões:

\begin{align}
u_m &= \frac{\Delta_{rb}}{\sqrt{3}} \quad \text{e} \quad u_V = \frac{\Delta_{rq}}{\sqrt{3}}.
\end{align}

Onde $\Delta_{rb}$ e $\Delta_{rq}$ representam as resoluções da balança e do béquer, respectivamente. A resolução da balança é \SI{1}{\gram} e a do béquer é \SI{5}{\milli\liter}, assim:

\begin{align}
u_m &= \frac{1.0}{\sqrt{3}} = 0.58 \, \text{g} \quad \text{e} \quad u_V = \frac{5.0}{\sqrt{3}} = 2.9 \, \text{cm}^3.
\end{align}

Desta maneira, encontramos:

\begin{align}
\rho_1 &= \frac{m}{V} = (10.00 \pm 0.97) \, \text{g/cm}^3.
\end{align}

Onde a incerteza associada a $\rho_1$, $u_{\rho_1}$, foi determinada utilizando a expressão a seguir considerando que a incerteza da balança é uma ordem de grandeza menor que a do béquer:

\begin{align}
u_{\rho_1} &= \frac{\overline{m}}{\overline{V}} \sqrt{\left(\frac{\sigma_V}{\overline{V}}\right)^2} = 0.97
\end{align}

\subsection{Modelo 2}

Para o segundo modelo experimental, encontramos os seguintes pares $m-m_a$ como resultados das medições:

\begin{align*}
(50.00 \pm 0.58) \, \text{g} & \quad (6.00 \pm 0.58) \, \text{g} \\
(100.00 \pm 0.58) \, \text{g} & \quad (12.00 \pm 0.58) \, \text{g} \\
(150.00 \pm 0.58) \, \text{g} & \quad (18.00 \pm 0.58) \, \text{g} \\
(200.00 \pm 0.58) \, \text{g} & \quad (24.00 \pm 0.58) \, \text{g} \\
(250.00 \pm 0.58) \, \text{g} & \quad (30.00 \pm 0.58) \, \text{g} \\
(300.00 \pm 0.58) \, \text{g} & \quad (36.00 \pm 0.58) \, \text{g}
\end{align*}

Os quais nos fornecem os seguintes resultados para a densidade $\rho_2$:

\begin{align*}
(7.97 \pm 0.78) \, \text{g/cm}^3 \\
(7.97 \pm 0.67) \, \text{g/cm}^3 \\
(7.97 \pm 0.27) \, \text{g/cm}^3 \\
(7.97 \pm 0.20) \, \text{g/cm}^3 \\
(7.97 \pm 0.16) \, \text{g/cm}^3 \\
(7.97 \pm 0.14) \, \text{g/cm}^3
\end{align*}

Onde a incerteza associada a cada valor encontrado para $\rho_2$, $u_{\rho_2}$, foi determinada utilizando a expressão:

\begin{align}
u_{\rho_2} &= \sqrt{\left(\frac{\rho_a}{m_a}\right)^2 u^2_m + \left(\frac{-m\rho_a}{m^2_a}\right)^2 u^2_{m_a}+\left(\frac{m}{m_a}\right)^2 u^2_{\rho_a}}.
\end{align}

Vale ressaltar que para todas as medições dos pares $m-ma$, o teste de flutuação de resultados indicou incertezas do tipo B, e assumindo distribuições de probabilidades retangulares para as medições, as incertezas foram calculadas de acordo com a expressão:

\begin{align}
u_m &= u_{ma} = \frac{\Delta_{rb}}{\sqrt{3}}.
\end{align}

Onde $\Delta_{rb}$ representa a resolução da balança utilizada. Como todos os valores encontrados para $\rho_2$ são compatíveis entre si, podemos combiná-los usando uma média ponderada para obter:

\begin{align}
\rho_2 &= \frac{m\rho_a}{m_a} = (8.333 \pm 0.086) \, \text{g/cm}^3.
\end{align}

\subsection{Modelo 3}

Para o terceiro modelo experimental, encontramos como resultados das medições:

\begin{align*}
m &= (50.00 \pm 0.58) \, \text{g} \\
h &= (7.800 \pm 0.029) \, \text{mm} \\
d_e &= (33.950 \pm 0.029) \, \text{mm} \\
d_i &= (3.400 \pm 0.029) \, \text{mm}
\end{align*}

Para todas as medições, o teste de flutuação de resultados indicou incertezas do tipo B. Assumindo distribuições de probabilidades retangulares para as medições, as incertezas foram calculadas de acordo com as expressões:

\begin{align}
u_m &= \frac{\Delta_{rb}}{\sqrt{3}}, \quad u_h = u_{de} = u_{di} = \frac{\Delta_{rp}}{\sqrt{3}}.
\end{align}

Onde $\Delta_{rb}$ and $\Delta_{rp}$ representam as resoluções da balança and do paquímetro, respectivamente. Desta maneira, encontramos:

\begin{align}
\rho_3 &= \frac{4m}{\pi (d^2_e - d^2_i)h} = (7.15 \pm 0.64) \, \text{g/cm}^3
\end{align}

\subsection{Análise de compatibilidade}

Finalmente, como os valores encontrados para $\rho_1$, $\rho_2$ e $\rho_3$ são compatíveis entre si, podemos combiná-los usando uma média ponderada para obter:

\begin{align}
\rho &= (8.309 \pm 0.085) \, \text{g/cm}^3
\end{align}

\section{Conclusões}
Com base na análise de dados, pode-se concluir que os três modelos experimentais foram satisfatórios para a determinação da densidade do cilindro, uma vez que os valores encontrados são compatíveis entre si. Ao combinarmos os valores obtidos e consultarmos a tabela de densidades fornecida na apostila, identificamos o material do cilindro como sendo latão recozido, $\rho_{\text{ref}} = 8.4$. Contudo, ao analisarmos individualmente cada resultado, encontramos erros relativos de 19.0%, 15.0% e 1.1% para cada modelo, respectivamente. Isso sugere maior efetividade do terceiro modelo, quando utilizamos instrumentos de medição com as resoluções descritas acima.

\end{multicols}

\end{document}
